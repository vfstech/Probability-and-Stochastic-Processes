
\let\negmedspace\undefined
\let\negthickspace\undefined
\documentclass[journal,12pt,twocolumn]{IEEEtran}

%\documentclass[conference]{IEEEtran}
%\IEEEoverridecommandlockouts
% The preceding line is only needed to identify funding in the first footnote. If that is unneeded, please comment it out.
\usepackage{cite}
\usepackage{amsmath,amssymb,amsfonts,amsthm}
\usepackage{amsmath}
\usepackage{algorithmic}
\usepackage{graphicx}
\usepackage{textcomp}
\usepackage{xcolor}
\usepackage{txfonts}
\usepackage{listings}
\usepackage{enumitem}
\usepackage{mathtools}
\usepackage{gensymb}
\usepackage[breaklinks=true]{hyperref}
\usepackage{tkz-euclide} % loads  TikZ and tkz-base
\usepackage{listings}
\usepackage{array}
\usepackage{mdframed}
\usepackage{listings}
\usepackage{xcolor}
\usepackage{mdframed}
\usepackage{siunitx}
\usepackage{graphicx}

\DeclareMathOperator*{\Res}{Res}
\renewcommand\thesection{\arabic{section}}
\renewcommand\thesubsection{\thesection.\arabic{subsection}}
\renewcommand\thesubsubsection{\thesubsection.\arabic{subsubsection}}

\renewcommand\thesectiondis{\arabic{section}}
\renewcommand\thesubsectiondis{\thesectiondis.\arabic{subsection}}
\renewcommand\thesubsubsectiondis{\thesubsectiondis.\arabic{subsubsection}}

% correct bad hyphenation here
\hyphenation{op-tical net-works semi-conduc-tor}
\def\inputGnumericTable{}                                 %%

\lstset{
%language=C,
frame=single, 
breaklines=true,
columns=fullflexible
}


% Define code style
\lstdefinestyle{mystyle}{
    language=Python,
    backgroundcolor=\color{black},
    basicstyle=\color{white}\ttfamily\footnotesize,
    keywordstyle=\color{purple},
    stringstyle=\color{orange},
    commentstyle=\color{gray},
    numbers=left,
    numberstyle=\tiny\color{gray},
    breakatwhitespace=false,
    breaklines=true,
    captionpos=b,
    keepspaces=true,
    showspaces=false,
    showstringspaces=false,
    showtabs=false,
    tabsize=4
}

% Define code environment with black background
\mdfdefinestyle{codebox}{
    backgroundcolor=black,
    linecolor=black,
    linewidth=0pt,
    innerleftmargin=3pt,
    innerrightmargin=3pt,
    innertopmargin=3pt,
    innerbottommargin=3pt
}

\begin{document}
%


\newtheorem{theorem}{Theorem}[section]
\newtheorem{problem}{Problem}
\newtheorem{proposition}{Proposition}[section]
\newtheorem{lemma}{Lemma}[section]
\newtheorem{corollary}[theorem]{Corollary}
\newtheorem{example}{Example}[section]
\newtheorem{definition}[problem]{Definition}

\newcommand{\BEQA}{\begin{eqnarray}}
\newcommand{\EEQA}{\end{eqnarray}}
\newcommand{\define}{\stackrel{\triangle}{=}}
\bibliographystyle{IEEEtran}

\providecommand{\mbf}{\mathbf}
\providecommand{\pr}[1]{\ensuremath{\Pr\left(#1\right)}}
\providecommand{\qfunc}[1]{\ensuremath{Q\left(#1\right)}}
\providecommand{\sbrak}[1]{\ensuremath{{}\left[#1\right]}}
\providecommand{\lsbrak}[1]{\ensuremath{{}\left[#1\right.}}
\providecommand{\rsbrak}[1]{\ensuremath{{}\left.#1\right]}}
\providecommand{\brak}[1]{\ensuremath{\left(#1\right)}}
\providecommand{\lbrak}[1]{\ensuremath{\left(#1\right.}}
\providecommand{\rbrak}[1]{\ensuremath{\left.#1\right)}}
\providecommand{\cbrak}[1]{\ensuremath{\left\{#1\right\}}}
\providecommand{\lcbrak}[1]{\ensuremath{\left\{#1\right.}}
\providecommand{\rcbrak}[1]{\ensuremath{\left.#1\right\}}}
\theoremstyle{remark}
\newtheorem{rem}{Remark}
\newcommand{\sgn}{\mathop{\mathrm{sgn}}}
\providecommand{\abs}[1]{\left\vert#1\right\vert}
\providecommand{\res}[1]{\Res\displaylimits_{#1}} 
\providecommand{\norm}[1]{\left\lVert#1\right\rVert}

\providecommand{\mtx}[1]{\mathbf{#1}}
\providecommand{\mean}[1]{E\left[ #1 \right]}
\providecommand{\fourier}{\overset{\mathcal{F}}{ \rightleftharpoons}}

\providecommand{\system}{\overset{\mathcal{H}}{ \longleftrightarrow}}
 
\newcommand{\solution}{\noindent \textbf{Solution: }}
\newcommand{\cosec}{\,\text{cosec}\,}
\providecommand{\dec}[2]{\ensuremath{\overset{#1}{\underset{#2}{\gtrless}}}}
\newcommand{\myvec}[1]{\ensuremath{\begin{pmatrix}#1\end{pmatrix}}}
\newcommand{\mydet}[1]{\ensuremath{\begin{vmatrix}#1\end{vmatrix}}}

\let\vec\mathbf

\vspace{3cm}
\title{

Probability Hardware Assignment

}
\author{ Vaibhav Falgun Shah (AI23MTECH02007)$^{}$
} 

\maketitle

\newpage
\bigskip
\renewcommand{\thefigure}{\theenumi}
\renewcommand{\thetable}{\theenumi}


\section{Introduction}
Task is to implement a random number generator using hardware components

\section{Components used}

\begin{table}[h]
\centering
\begin{tabular}{|c|c|}
\hline
\textbf{component} & \textbf{quantity} \\
\hline
Breadboard & 1 \\
\hline
Seven segment display & 1\\
\hline
Decoder & 1\\
\hline
D flip flop & 2\\
\hline
XOR Gate & 1\\
\hline
555 IC & 1 \\
\hline
Resistor 1K\si{\ohm}& 1 \\
\hline
Resistor 1M\si{\ohm} & 1 \\
\hline
Capacitor 100nF & 1 \\
\hline
Capacitor & 10nF \\
\hline
Jumper wires &  \\
\hline
\end{tabular}
\caption{components}
\end{table}
    
\section{Implementation}

\subsection{Random bit generation:}
\begin{itemize}
    \item Connect D1 of first flipflop to XOR Gate output
    \item Connect Q1 and Q2 outputs of both flipflops to their D inputs
    \item Connect output of XOR gate to D1 input of first flip flop
\end{itemize}

\subsection{Clock Signal generation:}
\begin{itemize}
    \item Connect 555 timer IC to generate clock pulses
    \item connect output of 555 IC to clock inputs of both flip flipflops
\end{itemize}

\subsection{7-segment display}
\begin{itemize}
    \item Connect Q1 and Q2 outputs of second flipflop to inputs of 7 segment Decoder
    \item Connect outputs of decoder to 7 segment display
\end{itemize}



\newpage

\section{Functionality}

\begin{itemize}
    \item As the power is applied, 555 timer IC generates clock pulses
    \item As these clock pulses are applied to flipflops, flip flops set their output according to their inputs
    \item XOR gate combines Q1 and Q2 outputs of first flipflops and random bits are generated
    \item These random bits are decoded by 7 segment decoder
    \item output of 7 segment decoder is fed into 7 segment display to display the randomly generated numbers generated by flipflops
\end{itemize}

\begin{figure}[h]
    \includegraphics[scale = 0.25]{/home/vaibhav/Downloads/msg-395638464-26408.jpg}
    \centering
    \graphicspath{ {/home/vaibhav/Downloads/msg-395638464-26408.jpg} }
    \caption{circuit generating random numbers}
\end{figure}

\end{document}