\let\negmedspace\undefined
\let\negthickspace\undefined
\documentclass[journal,12pt,twocolumn]{IEEEtran}

%\documentclass[conference]{IEEEtran}
%\IEEEoverridecommandlockouts
% The preceding line is only needed to identify funding in the first footnote. If that is unneeded, please comment it out.
\usepackage{cite}
\usepackage{amsmath,amssymb,amsfonts,amsthm}
\usepackage{amsmath}
\usepackage{algorithmic}
\usepackage{graphicx}
\usepackage{textcomp}
\usepackage{xcolor}
\usepackage{txfonts}
\usepackage{listings}
\usepackage{enumitem}
\usepackage{mathtools}
\usepackage{gensymb}
\usepackage[breaklinks=true]{hyperref}
\usepackage{tkz-euclide} % loads  TikZ and tkz-base
\usepackage{listings}
\usepackage{array}

\graphicspath{ {./images/} }

\DeclareMathOperator*{\Res}{Res}
\renewcommand\thesection{\arabic{section}}
\renewcommand\thesubsection{\thesection.\arabic{subsection}}
\renewcommand\thesubsubsection{\thesubsection.\arabic{subsubsection}}

\renewcommand\thesectiondis{\arabic{section}}
\renewcommand\thesubsectiondis{\thesectiondis.\arabic{subsection}}
\renewcommand\thesubsubsectiondis{\thesubsectiondis.\arabic{subsubsection}}

% correct bad hyphenation here
\hyphenation{op-tical net-works semi-conduc-tor}
\def\inputGnumericTable{}                                 %%

\lstset{
%language=C,
frame=single, 
breaklines=true,
columns=fullflexible
}


\begin{document}
%


\newtheorem{theorem}{Theorem}[section]
\newtheorem{problem}{Problem}
\newtheorem{proposition}{Proposition}[section]
\newtheorem{lemma}{Lemma}[section]
\newtheorem{corollary}[theorem]{Corollary}
\newtheorem{example}{Example}[section]
\newtheorem{definition}[problem]{Definition}

\newcommand{\BEQA}{\begin{eqnarray}}
\newcommand{\EEQA}{\end{eqnarray}}
\newcommand{\define}{\stackrel{\triangle}{=}}
\bibliographystyle{IEEEtran}

\providecommand{\mbf}{\mathbf}
\providecommand{\pr}[1]{\ensuremath{\Pr\left(#1\right)}}
\providecommand{\qfunc}[1]{\ensuremath{Q\left(#1\right)}}
\providecommand{\sbrak}[1]{\ensuremath{{}\left[#1\right]}}
\providecommand{\lsbrak}[1]{\ensuremath{{}\left[#1\right.}}
\providecommand{\rsbrak}[1]{\ensuremath{{}\left.#1\right]}}
\providecommand{\brak}[1]{\ensuremath{\left(#1\right)}}
\providecommand{\lbrak}[1]{\ensuremath{\left(#1\right.}}
\providecommand{\rbrak}[1]{\ensuremath{\left.#1\right)}}
\providecommand{\cbrak}[1]{\ensuremath{\left\{#1\right\}}}
\providecommand{\lcbrak}[1]{\ensuremath{\left\{#1\right.}}
\providecommand{\rcbrak}[1]{\ensuremath{\left.#1\right\}}}
\theoremstyle{remark}
\newtheorem{rem}{Remark}
\newcommand{\sgn}{\mathop{\mathrm{sgn}}}
\providecommand{\abs}[1]{\left\vert#1\right\vert}
\providecommand{\res}[1]{\Res\displaylimits_{#1}} 
\providecommand{\norm}[1]{\left\lVert#1\right\rVert}

\providecommand{\mtx}[1]{\mathbf{#1}}
\providecommand{\mean}[1]{E\left[ #1 \right]}
\providecommand{\fourier}{\overset{\mathcal{F}}{ \rightleftharpoons}}

\providecommand{\system}{\overset{\mathcal{H}}{ \longleftrightarrow}}
 
\newcommand{\solution}{\noindent \textbf{Solution: }}
\newcommand{\cosec}{\,\text{cosec}\,}
\providecommand{\dec}[2]{\ensuremath{\overset{#1}{\underset{#2}{\gtrless}}}}
\newcommand{\myvec}[1]{\ensuremath{\begin{pmatrix}#1\end{pmatrix}}}
\newcommand{\mydet}[1]{\ensuremath{\begin{vmatrix}#1\end{vmatrix}}}

\let\vec\mathbf

\vspace{3cm}
\title{

Probability Assignment - I

}
\author{ Vaibhav Falgun Shah (AI23MTECH02007)$^{}$
} 

\maketitle
\newpage

\bigskip
\renewcommand{\thefigure}{\theenumi}
\renewcommand{\thetable}{\theenumi}

\noindent\textbf{Question:}
\\ \\
Two dice are thrown simultaneously. If X denotes the number of sixes, find the expectation of X.
\\ \\
\noindent\textbf{Solution:}
\\ \\
Let X denote the random variable representing numer of sixes.
\\
Number of sixes when rolling two dice can be 0, 1 or 2.
\\

\noindent Hence, Sample space of X = \{0, 1, 2\} 
\\ \\
Expectation of X can be defined as:
\[ E(X) = 0 \cdot P(X=0) + 1 \cdot P(X=1) + 2 \cdot P(X=2)\]
\\
Now we have to find P(X=0), P(X=1) and P(X=2).
\\ \\
\noindent Let $p$ be the probability of getting a six on a single die, and $q$ be the probability of not getting a six on a single die.
\\
\noindent Since each die has 6 possible outcomes and only one of them is a six, we have $p = \frac{1}{6}$ and $q = 1 - p = \frac{5}{6}$.
\\ \\
\noindent We have two independent dice rolls, so the probability of getting no sixes, $P(X = 0)$, is $q^2$
\[ P(X = 0) = q \cdot q =  \left(\frac{5}{6}\right)^2 = \frac{25}{36} \]

\noindent We can get exactly one 6 in two ways: either we get six on first die and not on second, or we get six on second die and not on fist. So $P(X = 1)$ can be written as:
\[ P(X = 1) = (p \cdot q) + (q \cdot p) = 2 \cdot (p \cdot q) = 2 \cdot \left(\frac{1}{6} \cdot \frac{5}{6}\right) = \frac{10}{36} \]

\noindent The probability of getting two sixes, $P(X = 2)$, is $p^2$ since we have two independent dice:
\[ P(X = 2) = p \cdot p = \left(\frac{1}{6}\right)^2 = \frac{1}{36} \]

\noindent Now we have all 3 probabilities required to calculate the expectation:

\begin{table}[!htbp]
    \centering
    \begin{tabular}{|c|c|}
    \hline
    X & P(X) \\
    \hline
    \hline
    0 & $P(X = 0) = \dfrac{25}{36}$ \\
    \hline
    1 & $P(X = 1) = \dfrac{10}{36}$ \\
    \hline
    2 & $P(X = 2) = \dfrac{1}{36}$ \\
    \hline

    \end{tabular}
    %\caption{probabilities of X}
    \end{table}


\noindent Substituting these values back into the expectation formula:
\[ E(X) = 0 \cdot \left(\frac{25}{36}\right) + 1 \cdot \left(\frac{10}{36}\right) + 2 \cdot \left(\frac{1}{36}\right) = \frac{12}{36} = \frac{1}{3} \]

Therefore, the expectation of X is ${\dfrac{1}{3}}$

\end{document}